\documentclass[%
paper=a4,   % Papiergröße
size=12pt,  % Schriftgröße
draft=off   % Entwurfsmodus an/aus?
]{scrartcl}

% ----------------------------------------------------
% Essential packages
% ----------------------------------------------------
\usepackage[utf8]{inputenc}
\usepackage[T1]{fontenc}

% ----------------------------------------------------
% Fonts
% ----------------------------------------------------
\usepackage{lmodern}
\usepackage{roboto}

\setkomafont{subject}{\large\sffamily}
\addtokomafont{title}{\LARGE}
\addtokomafont{subtitle}{\large}
\addtokomafont{author}{\large}
\addtokomafont{publishers}{\large}

% ----------------------------------------------------
% Colors
% ----------------------------------------------------
\usepackage{graphicx}
\usepackage[svgnames]{xcolor}
\definecolor{darkgreen}{rgb}{0.23,0.46,0.23}

% Publication quality tables
\usepackage{booktabs}

% ----------------------------------------------------
% Commands for document metadata
% ----------------------------------------------------
\makeatletter
\newcommand{\email}[1]{\gdef\@email{#1}}
\newcommand{\matrno}[1]{\gdef\@matrno{#1}}
\newcommand{\institute}[1]{\gdef\@institute{#1}}
\makeatother

% Sprachauswahl:
%  main=* setzt die Hauptsprache für das Dokument
%  - ngerman --> deutsch
%  - english --> englisch
\def\languages{main=ngerman,english}
%\def\languages{main=english,ngerman}

% Informationen über den Autor und die Arbeit
\subject{%
  Seminar Software Engineering verteilter Systeme\\%
  Sommersemester 2017}

\title{Bio-inspired \& Grid Computing I}
\subtitle{What is it and why do we want it?}

% Name, Matrikelnummer und E-Mail-Adresse
\author{Max Mustermann}
\matrno{XXXXXXXX}
\email{mustermann@student.uni-augsburg.de}

% Datum der Abgabe
\date{30. Juni 2017}

\publishers{%
  Betreuer: Martina Musterfrau\\%
  Softwaremethodik für verteilte Systeme (Prof. Bauer)\\%
  Universität Augsburg}

%%% Local Variables:
%%% mode: latex
%%% TeX-master: "Seminararbeit"
%%% End:


% ----------------------------------------------------
% Multi-lingual documents with Babel
% ----------------------------------------------------
\usepackage{csquotes}
\usepackage[\languages]{babel}
\usepackage{iflang}

% ----------------------------------------------------
% Hyperlinks in PDF documents
% ----------------------------------------------------
\usepackage[%
colorlinks=true,        %
linkcolor=blue,         %
urlcolor=cyan,          %
citecolor=red,          %
raiselinks=true,        %
bookmarks=true,         %
bookmarksopenlevel=1,   %
bookmarksopen=true,     %
bookmarksnumbered=true, %
hyperindex=true,        % 
plainpages=false,       % correct hyperlinks
pdfpagelabels=true      % view TeX pagenumber in PDF reader
colorlinks=false,
citecolor=black,
]{hyperref} % erzeuge Hyperlinks z.B. für pdflatex

% Provides a solution to the problem with hyperref that links
% to floats actually anchor to the place below the float's caption,
% instead of anchoring to the beginning of the float
\usepackage[all]{hypcap}


% ----------------------------------------------------
% Code listings
% ----------------------------------------------------
\usepackage{listings}
\lstset{%
  basicstyle=\small\ttfamily,               % General font style for listings
  keywordstyle=\bfseries,                   % Font style for keywords
  commentstyle=\color{gray},                % Font style for comments
  stringstyle={},                           % Font style for string literals
  numbers=left,                             % Show line numbers
  stepnumber=1,                             % Step increments for line numbers
  numberstyle={\sffamily\tiny\color{gray}}, % Font style for line numbers
  numbersep=2em,                            % Space between line numbers and code
}

\pagestyle{plain}

% ----------------------------------------------------
% Bibliography management
% ----------------------------------------------------
\usepackage[%
backend=biber,
natbib=true,
citestyle=alphabetic,
bibstyle=alphabetic
]{biblatex}
\addbibresource{literature.bib}

% Use main body font for URLs in bibliography
\urlstyle{same}

% Intelligent cross-referencing
% Note: Must be loaded at end of preamble (esp. after hyperref)
\usepackage{cleveref}

%%% Local Variables:
%%% mode: latex
%%% TeX-master: "Seminararbeit"
%%% End:


\begin{document}
\IfLanguageName{ngerman}{\def\abstractname{Zusammenfassung}}{\def\abstractname{Abstract}}
  
\begin{titlepage}
  \makeatletter
  \phantom{Line}
  \vspace{2em}
  \begin{center}
    \ifdefempty{\@subject}{}{%
      {\usekomafont{subject}\@subject}
      \par\vspace{3em}
    }
    {\usekomafont{title}\@title}
    \ifdefempty{\@subtitle}{}{%
      \par\vspace{.5em}
      {\usekomafont{subtitle}\@subtitle}
    }
    \par\vspace{2em}
    {\usekomafont{author}%
      \@author\\
      Matrikelnummer: \@matrno\\
      \small\texttt{\@email}}
    \par\vspace{1.5em}
    {\usekomafont{publishers}%
      \@publishers}
  \end{center}
  \makeatother
  \begin{abstract}
    \noindent%
    \paragraph*{\abstractname}
    Lorem ipsum dolor sit amet, consectetuer adipiscing elit, sed diam nonummy nibh euismod tincidunt ut laoreet dolore magna aliquam erat volutpat. Ut wisi enim ad minim veniam, quis nostrud exerci tation ullamcorper suscipit lobortis nisl ut aliquip ex ea commodo consequat.
  \end{abstract}

  \vfill
  \centering
  \includegraphics[height=40mm]{figures/uni_siegel}
  \vspace{2em}
\end{titlepage}
%%% Local Variables:
%%% mode: latex
%%% TeX-master: "Seminararbeit"
%%% End:


% Inhaltsverzeichnis ist nur für die Besprechungen.
% Für die endgültige Abgabeversion auskommentieren.
\tableofcontents

\clearpage
\section{Einleitung}
\label{sec:Einleitung}
Lorem ipsum dolor sit amet, consectetuer adipiscing elit, sed diam nonummy nibh euismod tincidunt ut laoreet dolore magna aliquam erat volutpat. Ut wisi enim ad minim veniam, quis nostrud exerci tation ullamcorper suscipit lobortis nisl ut aliquip ex ea commodo consequat. \cite{Konak2006,Sailer2013}

Lorem ipsum dolor sit amet, consectetuer adipiscing elit, sed diam nonummy nibh euismod tincidunt ut laoreet dolore magna aliquam erat volutpat. Ut wisi enim ad minim veniam, quis nostrud exerci tation ullamcorper suscipit lobortis nisl ut aliquip ex ea commodo consequat.

\section{Hauptteil}
\label{sec:Hauptteil}

Abbildungen können im Unterverzeichnis \texttt{figures} abgelegt werden.
Eingebunden werden Sie mit dem Befehl \texttt{\textbackslash includegraphics} innerhalb
einer \texttt{figure}-Umgebung:
\begin{figure}[htb]
  \centering
  \includegraphics[width=0.8\textwidth]{figures/mda_transformations.png}
  \caption{MDA-Transformationen (Abbildung~5.19 in \cite{sommerville})}
  \label{fig:power}
\end{figure}

\subsection{Erste Zwischenüberschrift}
\label{sec:ErsteZwischenueberschrift}
Die Arbeit kann auch Tabellen im \texttt{table}-Environment enthalten:
\begin{table}[ht]
  \centering
  \caption{Entfernungstabelle Süddeutschland, vgl. \cite{entfernungstabelle}}
  \begin{tabular}{c r r r}
    \toprule
              & Augsburg & München & Stuttgart \\
    \midrule
    Augsburg  & -        & 61      & 149       \\
    München   & 61       & -       & 210       \\
    Stuttgart & 149      & 210     & -         \\
    \bottomrule
  \end{tabular}
  \label{tab:entfernungen}
\end{table}

\subsubsection{Erste Unterüberschrift}
\label{sec:ErsteUnterueberschrift}

Das \texttt{listings}-Paket erlaubt es, Quellcode mit Syntax-Highlighting einzubinden:

\begin{lstlisting}[language=Python,float=ht,caption={Python-Programm zur Berechnung der Fakultätsfunktion}]
def fact(n):
    """Return the n-th factorial number"""
    if n == 0:
        return 1
    else:
        return n * fact(n-1)
  
# Test output
print fact(10)
print "Done"
\end{lstlisting}
\label{lst:factorial}  
  
\subsection{Zweite Zwischenüberschrift}
\label{sec:ZweiteZwischenueberschrift}

TEXT

\section{Schluss}
\label{sec:Schluss}

TEXT

% Literaturverzeichnis
\printbibliography[heading=bibintoc]

% Anhang
\include{appendix}

% Eidesstattliche Erklärung
\clearpage
\section{Eidesstattliche Erklärung}
Ich versichere, dass ich die vorliegende Arbeit ohne fremde Hilfe und ohne Benutzung anderer
als der angegebenen Quellen angefertigt habe, und dass die Arbeit in gleicher oder ähnlicher
Form noch keiner anderen Prüfungsbehörde vorgelegen hat.

Alle Ausführungen der Arbeit, die wörtlich oder sinngemäß übernommen wurden, sind als solche
gekennzeichnet.

\vspace{1em}

\makeatletter
[\@author]
[Augsburg, den \@date]
\makeatother

%%% Local Variables:
%%% mode: latex
%%% TeX-master: "Seminararbeit"
%%% End:


\end{document}